\chapter{Design of cached}\label{ch:cached-design}

\section{General}

Currently, Archipelago doesn't have an xseg peer that can act as a cache for
block requests. Having a cache however is essential for a tiered storage and
cached peer is implemented for this reason.

\section{Implementation details}

cached is based on xseg and also on the following xtypes:

\begin{itemize}
	\item \textbf{xcache} (for cache support)
	* xwork (for job support)
	* xworkq (for atomicity in execution of jobs)
	* xwaitq (for conditional execution of jobs)
\end{itemize}

More specifically, cached consists of the cache provided by xcache and a
pre-allocated number of objects. An object is divided in buckets and its size,
as well as bucket size, are defined by the user.

The fact that objects are pre-allocated means two things:

1) We don't need to care about memory fragmentation and system call overhead
2) We cannot index single buckets. <FILLME>

\section{The xcache xtype}



\section{The xworkq xtype}

Every object has a workq. Whenever a new request is accepted/received for an
object, it is enqueued in the workq and we are sure that only one thread at a
time can have access to the objects data and metadata.

For more information, see the xworkq.

\section{The xwaitq xtype}

When a thread tries to insert an object in cache but fails, due to the fact
that cache is full, the request is enqueued in the xcache waitq, which is
signaled every time an object is freed.

For more information, see the xwaitq.

\section{Cached internals}

\subsection{Object states}

Every object has a state, which is set atomically by threads. The state list is
the following:

* READY: the object is ready to be used
* FLUSHING: the object is flushing its dirty buckets
* DELETING: there is a delete request that has been sent to the blocker for
  this object
* INVALIDATED: the object has been deleted
* FAILED: something went very wrong with this object

Also, object buckets have their own states too:

* INVALID: the same as empty
* LOADING: there is a pending read to blocker for this bucket
* VALID: the bucket is clean and can be read
* DIRTY: the bucket can be read but its contents have not been written to the
  underlying storage
* WRITING: there is a pending write to blocker for this bucket

Finally, for every object there are bucket state counters, which are increased/
decreased when a bucket state is changed. These counters give us an O(1)
glimpse to the bucket states of an object.

\subsection{Per-object peer requests}

Reads and writes to objects are practically read/write request from other
peers, for which a peer request has been allocated. There are cases though
when an object has to allocate its own peer request e.g. due to a flushing of
its dirty buckets. Since this must be fast, there are pre-allocated requests
hard-coded in the struct of each object which can be used in such cases.

\subsection{Write policy}

The user must define beforehand what is the write policy of cache. There are
two options: writethrough and writeback. On a side note, as far as reads and
cache misses are concerned, cached operates under a write-allocate policy.

